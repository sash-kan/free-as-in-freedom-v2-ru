\chapter{Приложение Б -- Лицензия GNU для свободно используемой
  документации}
\label{Appendix B}

\phantomsection  % so hyperref creates bookmarks

 \begin{center}
  Это неофициальный перевод лицензии GNU для свободно используемой
  документации (GNU Free Documentation License) на русский язык. Он
  был опубликован не Фондом свободного программного обеспечения и не
  содержит условий распространения текстов, которые используют GNU FDL
  -- для этого пригоден только ее исходный английский текст. Тем не
  менее, мы надеемся, что этот перевод поможет лучше ее понять.

  Вы можете распространять перевод, с изменениями или без, только с
  соблюдением условий, описанных по адресу:
  \href{http://www.gnu.org/licenses/translations.html}{gnu.org/licenses/translations.html}.
 \end{center}

 \begin{center}

       Version 1.3, 3 November 2008


 Copyright \copyright{} 2000, 2001, 2002, 2007, 2008  Free Software
 Foundation, Inc. Перевод: Павел Протасов <pvphome@gmail.com>, 2016 г.

 \bigskip

     \url{http://fsf.org/}

 \bigskip

 Разрешается свободно копировать и распространять текст настоящей
 Лицензии, запрещается вносить в него изменения.
\end{center}


\begin{center}
{\bf\large ВВЕДЕНИЕ}
\end{center}

Настоящая Лицензия предназначена для того, чтобы сделать руководство,
учебник, другой технический текст или инструкцию \enquote{свободными},
то есть свободно используемыми, для того чтобы гарантировать всем
право свободно создавать копии и распространять их, с изменениями или
без, как для извлечения прибыли, так и в некоммерческих целях. Кроме
того, настоящая Лицензия сохраняет за автором и издателем возможность
защиты репутации, не позволяя возложить на них ответственность за
изменения текста, сделанные другими.

Настоящая Лицензия является разновидностью так называемого
\enquote{копилефта}; это означает, что производные произведения,
созданные на основе документа, должны быть \enquote{свободными} в том
же смысле. Она дополняет Генеральную публичную лицензию GNU, которая
является \enquote{копилефтной} лицензией, предназначенной для
свободного программного обеспечения.

Настоящая Лицензия предназначена для использования с руководствами для
свободных программ, поскольку к свободному программному обеспечению
должна прилагаться свободная документация: свободное программное
обеспечение должно распространяться с руководствами, которые можно
использовать на тех же условиях, что и само программное
обеспечение. Но сфера применения настоящей Лицензии не ограничивается
руководствами к программному обеспечению; она может использоваться с
любыми текстовыми произведениями независимо от их тематики и того,
опубликованы ли они в печатном виде. Настоящую Лицензию рекомендуется
использовать для произведений, назначением которых является обучение
или предоставление справочной информации.

\begin{center}
{\Large\bf 1. СФЕРА ПРИМЕНЕНИЯ И ОПРЕДЕЛЕНИЯ\par}
\phantomsection
\end{center}

Настоящая Лицензия применяется к любым руководствам или другим
произведениям, существующим в любой форме и имеющим уведомление о том,
что они могут быть использованы в соответствии с настоящей Лицензией,
сделанное обладателем авторских прав. При помощи такого уведомления
предоставляется право на использование произведения на описанных
условиях на территории всего мира, без выплаты авторских отчислений, в
течение неограниченного времени. Термин \enquote{Документ} далее
означает любое такое руководство или произведение. Любое третье лицо
считается лицензиатом и именуется \enquote{Вы}. Вы принимаете условия
данной лицензии, если создаете копию, вносите изменения или
распространяете Произведение способом, на который требуется разрешение
в соответствии с авторским правом.

\enquote{Измененной версией} Документа называется любое произведение,
содержащее Документ или его часть, скопированную дословно либо с
изменениями и (или) переведенную на другой язык.

\enquote{Второстепенным разделом} называется озаглавленное приложение
или титульный раздел Документа, имеющее отношение к тематике Документа
в целом (или связанным с ней вопросам) исключительно в связи с
издателями или авторами Документа и не содержащее ничего относящегося
к данной тематике напрямую. (Таким образом, если Документ
математический, его Второстепенный раздел не может иметь отношения к
математике.) Такая связь с тематикой документа или смежными вопросами
может сложиться исторически либо отражать правовую, коммерческую,
философскую, этическую или политическую позицию в отношении них.

\enquote{Неизменяемыми разделами} называются Второстепенные разделы,
названия которых отнесены к Неизменяемым в уведомлении, в котором
сообщается о том, что Документ распространяется на условиях настоящей
Лицензии. Если раздел не удовлетворяет приведенному выше определению
Второстепенного раздела, он не может быть включен в состав
Неизменяемых. Документ может не содержать Неизменяемых разделов. Если
Документ не содержит указания на наличие Неизменяемых разделов, это
означает, что такие разделы отсутствуют.

\enquote{Текстами обложки} называются короткие отрывки текста, которые
отнесены к Текстам первой или задней страницы обложки в уведомлении, в
котором сообщается о том, что Документ распространяется на условиях
настоящей Лицензии. Текст первой страницы обложки может включать в
себя не более 5 слов, Текст задней страницы обложки может включать в
себя не более 25 слов.

\enquote{Открытой} копией Документа называется машиночитаемая копия,
представленная в формате, имеющем общедоступное описание, который
пригоден для непосредственного изменения документа при помощи
текстовых редакторов общего назначения или (для растровых изображений)
программ для рисования общего назначения или (для рисунков)
распространенными графическими редакторами, а также пригоден для
работы при помощи программ форматирования текста или автоматического
конвертирования в форматы, пригодные для работы при помощи программ
форматирования текста. Копия, созданная в \enquote{Открытом} формате,
разметка или отсутствие разметки которой препятствует или затрудняет
внесение в нее изменений в будущем, не является
\enquote{Открытой}. Формат для представления изображения не является
\enquote{Открытым} в том случае, если он используется для
представления любого значимого отрывка текста. Копия, представленная
не в Открытом формате, называется \enquote{Закрытой}.

Примерный перечень форматов, пригодных для создания Открытых копий,
включает в себя простой текст ASCII без разметки, входные форматы
Texinfo и LaTeX, SGML или XML, использующие общедоступные описания, а
также соответствующий стандартам простой HTML, PostScript или PDF,
предназначенный для внесения изменений человеком. Примерный перечень
открытых форматов изображений включает в себя PNG, XCF и
JPG. Закрытыми являются такие форматы, использование которых
ограничено, которые могут быть прочитаны и отредактированы только
предназначенными для этого текстовыми процессорами, а также SGML или
XML, для которых отсутствуют общедоступные описания и (или) средства
обработки.

\enquote{Титульным листом} называется, для печатных книг, сам
титульный лист, а также страницы, необходимые для изложения
материалов, которые, в соответствии с настоящей Лицензией, должны
содержаться на титульном листе. Для произведений в форматах, не
содержащих титульного листа как такового, \enquote{Титульным листом}
считается текст рядом с наиболее выделенным заглавием произведения,
предшествующим основному тексту.

\enquote{Издателем} называется любое физическое или юридическое лицо,
которое распространяет копии Документа для неограниченного круга
лиц. Разделом, \enquote{Озаглавленным XYZ}, называется раздел,
заголовок которого содержит последовательность \enquote{XYZ} либо
последовательность \enquote{XYZ} содержится в скобках после текста
перевода \enquote{XYZ} на другой язык.

(В данном случае \enquote{XYZ} обозначает конкретное название раздела,
упомянутое ниже, например,\enquote{Заявления}, \enquote{Посвящения},
\enquote{Благодарности} или \enquote{История}. \enquote{Сохранение
  названия} такого раздела при изменении Документа означает, что
раздел остается \enquote{Озаглавленным XYZ} в соответствии с данным
определением.

Документ может включать текст Отказа от ответственности после
уведомления о том, что к нему применима настоящая Лицензия. Этот Отказ
от ответственности считается включенным в настоящую Лицензию при
помощи ссылки на него, но только в отношении предоставляемых
полномочий: любые другие толкования, которые может иметь данный Отказ
от ответственности, являются недействительными и не имеют никакого
значения для настоящей Лицензии.


\begin{center}
{\Large\bf 2. КОПИРОВАНИЕ В НЕИЗМЕННОМ ВИДЕ\par}
\phantomsection
\end{center}

Вам разрешается копировать и распространять Документ на любом
носителе, как для извлечения прибыли, так и в некоммерческих целях,
при условии, что ко всем его экземплярам прилагаются: настоящая
Лицензия, информация об авторских правах, а также уведомление о том,
что Документ используется на условиях настоящей Лицензии
и Вы не добавляли к настоящей Лицензии других условий. Вы не можете
использовать технические средства для затруднения или контроля чтения
или копирования тех материалов, которые Вы изготовляете или
распространяете. Однако Вы можете запрашивать компенсацию за
копии. Если Вы распространяете большое количество копий, Вы должны
также соблюдать условия, изложенные в разделе 3.

Вы можете временно предоставлять копии на условиях, изложенных выше, а
также производить их публичный показ.

\begin{center}
{\Large\bf 3. МАССОВОЕ СОЗДАНИЕ КОПИЙ\par}
\phantomsection
\end{center}

Если Вы публикуете Документ в виде печатных копий (или копий в
изданиях, которые обычно имеют печатную обложку), в количестве более
100 экземпляров, и в уведомлении о том, что Документ используется на
условиях настоящей Лицензии, говорится о наличии Текстов обложки, Вы
должны снабдить копии обложкой, которая четко и в явной форме содержит
следующие Тексты обложки: Тексты Первой страницы обложки на первой
странице и Тексты задней страницы обложки на последней странице. На
обеих страницах обложки Вы также должны четко и в явной форме указать
на то, что являетесь издателем этих копий. Первая страница должна
включать полное название, со всеми словами, составляющими название,
выделенными и видными одинаково. Вы также можете добавить на обложку
другие материалы. В случае, если изменения при создании копии
затрагивают только тексты обложек и при этом сохраняется название
документа, такое копирование может во всем остальном рассматриваться
как копирование в неизменном виде.

Если текст, который должен содержаться на любой из страниц обложки,
слишком длинный, чтобы поместиться на обложку, Вы должны поместить его
начало (в том объеме, который является приемлемым) на соответствующую
страницу, а продолжение разместить на следующих страницах.

Если Вы распространяете Закрытые копии Документа в количестве,
превышающем 100 экземпляров, Вы должны прилагать машинночитаемую
Открытую копию к каждой Закрытой копии либо указать в каждой копии или
приложить к ней адрес в сети, с которого любой пользователь может
получить доступ к Открытой копии Документа, свободной от добавленных
материалов, при помощи обычного способа получения данных по сети. Если
Вы выбрали последний вариант, при начале массового распространения
Закрытых копий Вы должны предпринять разумные шаги, чтобы убедиться в
том, что Открытую копию можно получить по указанному адресу в течение
как минимум одного года после распространения последней Закрытой копии
этого издания (от Вас, Ваших агентов или распространителей).

Желательно, но не обязательно, связаться с авторами Документа перед
массовым распространением его копий, чтобы дать им возможность
предоставить Вам обновленную версию Документа.

\begin{center}
{\Large\bf 4. ВНЕСЕНИЕ ИЗМЕНЕНИЙ\par}
\phantomsection
\end{center}

Если Вы копируете или распространяете Измененную версию Документа на
условиях приведенных выше разделов 2 и 3, считается, что Вы
распространяете Измененную версию именно в соответствии с настоящей
Лицензией; Измененная версия при этом считается Документом, право на
ее распространение и изменение передается любому обладателю ее
копии. Кроме того, Вы должны выполнить следующие действия с Измененной
версией:

\begin{itemize}
\item[A.]
Поместить на Титуальном листе (и на обложке, если она есть) название,
отличающееся от названий Документа и его предыдущих редакций (которые,
при наличии, должны быть перечислены в разделе «История»
Документа). Вы можете использовать название предыдущей редакции
документа в случае, если получили разрешение от издателя этой
редакции.

\item[B.]
Перечислить на Титульном листе в качестве авторов имена физического
или юридического лица или лиц, являющихся авторами изменений
Измененной версии, наряду с именами минимум пяти авторов исходного
Документа (всех авторов, если их менее пяти), в случае, если они не
освободили Вас от выполнения этого требования.

\item[C.]
Привести на Заглавной страниaце имя издателя Измененной версии, указав
на то, что он является издателем.

\item[D.]
Сохранить все уведомления об авторском праве, имеющиеся в Документе.

\item[E.]
Добавить уведомление об авторском праве на Ваши изменения к другим
уведомлениям об авторских правах.

\item[F.]
Включить в текст сразу же после уведомлений об авторских правах
уведомление о лицензии, по форме, приведенной в Приложении ниже,
которым пользователям дается разрешение на использование Измененной
версии на условиях настоящей Лицензии.

\item[G.]
Сохранить в уведомлении о лицензии полные списки Неизменяемых разделов
и необходимых Текстов обложки, приведенных в лицензии Документа.

\item[H.]
Приложить точную копию настоящей Лицензии.

\item[I.]
Сохранить раздел, озаглавленный \enquote{История}, включая его
Заголовок, добавив в его конце как минимум название, год выпуска,
новых авторов и издателя Измененной версии в том виде, в котором они
приведены на Заглавной странице.Если в Документе отсутствует раздел,
озаглавленный \enquote{История}, добавьте его, включив туда название,
год выпуска, авторов и издателя Документа в том виде, в котором они
приведены на Заглавной странице, после чего добавьте к нему сведения
об Измененной версии так, как это описано в предыдущем предложении.

\item[J.]
Сохраните сетевой адрес, если он указан в Документе, по которому можно
получить Открытую копию Документа, а также приведенные в нем сетевые
адреса для получения предыдущих редакций Документа, на которых он
основан. Они могут быть приведены в разделе \enquote{История}. Вы
можете не включать сетевой адрес для произведения, которое либо было
опубликовано более чем за четыре года до публикации Документа, либо в
том случае, если получили разрешение на это у издателя предыдущей
редакции.

\item[K.]
Для любого раздела, озаглавленного \enquote{Благодарности} или
\enquote{Посвящения}, сохраните Заголовок раздела, а также содержание
и стиль каждого подтверждения и (или) посвящения, касающегося
соавтора.

\item[L.]
Сохраните все Неизменяемые разделы Документа, воспроизведя их текст и
заголовки в том же виде. Номера разделов или их эквивалент не
считаются частью заголовков разделов.

\item[M.]
Удалите любой раздел, озаглавленный \enquote{Одобрения}. Такие разделы
не могут включаться в Измененную версию.

\item[N.]
Не переименовывайте ни один из разделов так, чтобы он назывался
\enquote{Одобрения} или его заголовок совпадал с заголовком любого
Неизменяемого раздела.

\item[O.]
Сохраните все Отказы от гарантий.
\end{itemize}

Если Измененная версия включает в себя предисловия или приложения,
удовлетворяющие определению Второстепенных разделов и не содержащие
материалов, скопированных из Документа, Вы можете по желанию назвать
Неизменяемыми некоторые из этих разделов или все их. Чтобы это
сделать, добавьте их заголовки к списку Неизменяемых разделов в
уведомление о том, что Измененная версия используется на условиях
настоящей Лицензии. Эти заголовки должны отличаться от заголовков
других разделов.

Вы можете добавить раздел, озаглавленный \enquote{Одобрения},
содержащий только сведения об одобрении Вашей Измененной версии
различными лицами — например, указание на то, что она подверглась
рецензированию или о том, что текст был одобрен организацией как
официальное определение стандарта.

Вы можете добавить отрывок длиной до пяти слов в качестве Текста
первой страницы обложки и отрывок длиной до 25 слов в качестве Текста
задней страницы обложки, в конце перечисления Текстов обложки в
Измененной версии.

Любым лицом может быть добавлен (лично или по поручению) только один
отрывок Текста первой страницы обложки и один — Текста задней страницы
обложки. Если Документ уже содержит текст на той же странице обложки,
добавленный ранее Вами или по поручению любого лица, в интересах
которого Вы действуете, Вы не можете добавлять еще один, но Вы можете
заместить старый текст, при наличии разрешения от предыдущего
издателя, который его добавил, данного в явной форме.

Автор (авторы) и издатель (издатели) Документа не передают по
настоящей Лицензии разрешения на использование своих имен для рекламы
либо заявлений или уведомлений об одобрении Измененной версии.

\begin{center}
{\Large\bf 5. ОБЪЕДИНЕНИЕ ДОКУМЕНТОВ\par}
\phantomsection
\end{center}

Вы можете объединять Документ с другими документами, опубликованными
на условиях настоящей Лицензии, соблюдая условия распространения
измененных версий, описанные в разделе 4, включив в подборку в
неизменном виде все Неизменяемые разделы всех первоначальных
документов, указав в уведомлении об условиях использования данного
произведения на то, что все они являются его Неизменяемыми разделами,
а также сохраняя все условия об отказе от ответственности.

Составное произведение может содержать только одну копию настоящей
Лицензии, несколько одинаковых Неизменяемых разделов могут быть
заменены одной копией. Если существует несколько Неизменяемых разделов
с одинаковыми именами, но разным содержанием, необходимо сделать
название каждого из них уникальным путем добавления в конце него в
скобках имени первоначального автора или издателя данного раздела,
если оно известно, либо уникального номера. Отразите изменение
названия раздела в списке Неизменяемых разделов в уведомлении об
условиях использования составного произведения.

При объединении Вы должны объединить все разделы \enquote{История}
исходных документов в один общий раздел \enquote{История}; также нужно
объединить все разделы \enquote{Благодарности}, и разделы под
названием \enquote{Посвящения}. Вы должны удалить все разделы под
названием \enquote{Одобрения}.

\begin{center}
{\Large\bf 6. СБОРНИКИ ДОКУМЕНТОВ\par}
\phantomsection
\end{center}

Вы можете создать сборник, состоящий из Документа и других документов,
выпущенных на условиях настоящей Лицензии, заменив разные копии
настоящей Лицензии из разных документов на один экземпляр, включенный
в сборник, при условии, что во всем остальном для каждого из
документов Вы выполнили требования настоящей Лицензии для копирования
в неизменном виде.

Вы можете изъять документ из сборника и распространять его отдельно в
соответствии с настоящей Лицензией, прилагая к документу копию
настоящей Лицензии и соблюдая для этого документа ее требования во
всем остальном, что касается копирования в неизменном виде.

\begin{center}
{\Large\bf 7. ОБЪЕДИНЕНИЕ С ДРУГИМИ ПРОИЗВЕДЕНИЯМИ\par}
\phantomsection
\end{center}

Соединение Документа или производных от него произведений с другими
отдельными и независимыми документами или произведениями, на носителе,
предназначенном для хранения или распространения информации,
называется \enquote{набором}, если авторское право на получившуюся в
результате подборку не используется для ограничения прав пользователей
компиляции дополнительно к тому, что требуют разрешения на отдельные
произведения. Если Документ включается в набор, настоящая Лицензия не
распространяется на другие работы в наборе, не являющиеся производными
от Документа.

Если требования к Текстам обложки, приведенные в разделе 3, применимы
к этим копиям Документа и если величина документа не превышает
половины всего набора, Тексты обложки Документа могут быть помещены на
страницах обложки, которой снабжен набор документов, или электронный
эквивалент таких страниц, если Документ находится в электронном
виде. В противном случае они должны быть размещены на печатных
страницах обложки, которой снабжен весь набор.

\begin{center}
{\Large\bf 8. ПЕРЕВОД\par}
\phantomsection
\end{center}

Перевод считается разновидностью внесения изменений, поэтому Вы можете
распространять переводы Документа в соответствии с положениями раздела
4. Замена Неизменяемых разделов на их переводы требует специального
разрешения от обладателей авторских прав на них, но Вы можете включить
в документ переводы некоторых или всех Неизменяемых разделов вместе с
их исходными версиями. Вы можете включить в Документ перевод настоящей
Лицензии и всех уведомлений об условиях использования, а также любых
отказов от ответственности, при условии, что Вы также включаете в него
исходную английскую версию настоящей Лицензии и исходные версии этих
уведомлений и отказов. В случае расхождений между переводом и исходной
версией настоящей Лицензии, уведомлений или отказов от
ответственности, верными считаются исходные версии.

Если существует раздел Документа под названием
\enquote{Благодарности}, \enquote{Посвящения} или \enquote{История},
для того, чтобы выполнить требование (раздел 4) Сохранить его название
(раздел 1), как правило, требуется изменить их названия.

\begin{center}
{\Large\bf 9. ПРЕКРАЩЕНИЕ ДЕЙСТВИЯ\par}
\phantomsection
\end{center}

Вы не можете копировать, изменять, осуществлять сублицензирование
Документа или распространять его, за исключением случаев, специально
оговоренных в условиях настоящей Лицензии. Любая такая попытка
копировать, изменять, сублицензировать или распространять его является
ничтожной и автоматически прекращает Ваши права, переданные по
настоящей Лицензии.

Тем не менее, если Вы прекращаете нарушение настоящей Лицензии, Ваши
права, полученные от конкретного правообладателя, восстанавливаются
(a) временно, до тех пор пока правообладатель явно и окончательно не
прекратит действие Ваших прав, и (b) навсегда, если правообладатель не
уведомит Вас о нарушении с помощью надлежащих средств в течение 60
дней после прекращения нарушений.

Кроме того, Ваши права, полученные от конкретного правообладателя,
восстанавливаются навсегда, если правообладатель впервые любым
подходящим способом уведомляет Вас о нарушении настоящей Лицензии на
свое произведение (для любого произведения) и Вы устраняете нарушение
в течение 30 дней после получения уведомления.

Прекращение Ваших прав, описанное в настоящем разделе, не прекращает
действие лицензий лиц, которые получили от Вас копии произведения или
права, предоставляемые настоящей Лицензией. Если Ваши права были
прекращены и не восстановлены на постоянной основе, получение полной
или частичной копии тех же материалов не дает Вам никаких прав на их
использование.

\begin{center}
{\Large\bf 10. ПЕРЕСМОТР УСЛОВИЙ НАСТОЯЩЕЙ ЛИЦЕНЗИИ\par}
\phantomsection
\end{center}

Фонд свободного программного обеспечения время от времени может
публиковать пересмотренные и (или) новые редакции Лицензии GNU для
свободно используемой документации. Они будут аналогичны по смыслу
настоящей редакции, но могут отличаться от нее в деталях, направленных
на решение новых проблем или регулирование новых отношений. См.
\href{https://www.gnu.org/copyleft/}{gnu.org/copyleft/}.

Каждой редакции присваивается собственный номер. Если в Документе
указано, что он распространяется на условиях определенной версии
настоящей Лицензии \enquote{или любой более поздней версии}, Вы можете
пользоваться терминами и условиями этой или более поздней редакции,
которая была опубликована Фондом свободного программного обеспечения
(за исключением черновых версий). Если в Документе не указан номер
редакции Лицензии GNU для свободно используемой документации, Вы
можете выбрать любую редакцию, опубликованную Фондом свободного
программного обеспечения. Если в Документе указано, что лицо,
осуществляющее передачу, может выбрать, какую из будущих редакций
Лицензии GNU для свободно используемой документации использовать,
публичное заявление такого лица о принятии редакции дает Вам право
окончательно выбрать эту редакцию для Программы.

\begin{center}
{\Large\bf 11. ПОВТОРНОЕ ЛИЦЕНЗИРОВАНИЕ\par}
\phantomsection
\end{center}

\enquote{Многопользовательский сайт для совместной работы} (или
\enquote{MCР-сайт}) означает любой интернет-сервер, который публикует
охраноспособные произведения, а также предоставляет пользователю
развитые возможности для редактирования этих
произведений. Общедоступный вики-сайт, статьи которого редактировать
может каждый посетитель, является примером такого
сервера. \enquote{Многопользовательская совместная работа} (или
\enquote{МСР}), содержащаяся на сайте, означает набор охраноспособных
произведений, опубликованных на МСР-сайте таким способом.

Термин \enquote{CC-BY-SA} означает лицензию \enquote{Creative Commons
  Attribution-Share Alike 3.0}, опубликованную некоммерческой
компанией Creative Commons Corporation, расположенной в Сан-Франциско,
штат Калифорния, а также будущие редакции данной лицензии,
опубликованные этой же организацией и имеющие условия о том, чтобы
производные произведения свободно использовались на условиях этой же
лицензии.

\enquote{Включение} означает публикацию или переиздание всего
Документа или его части в качестве части другого Документа.

МСР \enquote{пригодна для повторного лицензирования}, если она
распространяется на условиях настоящей Лицензии и если все
произведения, которые были впервые опубликованы на условиях настоящей
Лицензии где-то, кроме этой МСР, а затем включены полностью или
частично в МСР, (1) не имеют Текстов обложки или Неизменяемых разделов
и (2) были включены таким способом в МСР до 1 ноября 2008 года.

Оператор МСР-сайта может повторно опубликовать содержащуюся на
МСР-сайте на условиях CC-BY-SA на том же сайте в любое время до 1
августа 2009 года, при условии, что МСР пригодна для повторного
лицензирования.

\begin{center}
{\Large\bf ПРИЛОЖЕНИЕ: Как применять настоящую Лицензию к Вашим документам\par}
\phantomsection
\end{center}

Для того, чтобы распространить условия настоящей Лицензии на документ,
который Вы написали, включите копию Лицензии в документ и поместите
следующее уведомление об авторских правах и условиях использования
непосредственно после титульного листа:

\bigskip
\begin{quote}
Copyright \copyright{} ГОД ВАШЕ ИМЯ. Разрешается копировать,
распространять и (или) изменять этот документ в соответствии с
условиями редакции 1.3 Лицензии GNU для свободно используемой
документации или более поздней редакции, опубликованной Фондом
свободного программного обеспечения; при отсутствии Неизменяемых
разделов, Текстов первой и задней страницы обложки. Копия Лицензии
включена в раздел, озаглавленный \enquote{GNU Free Documentation
License}.
\end{quote}
\bigskip

Если у Вас имеются Неизменяемые разделы, Тексты первой и задней
страницы обложки, замените слова \enquote{при отсутствии \dots задней
страницы обложки} следующей строкой:

\bigskip
\begin{quote}
Неизменяемыми разделами являются [перечислите их названия], а также
Тексты первой страницы обложки [перечислите], и Тексты задней страницы
обложки [перечислите].
\end{quote}
\bigskip

Если у Вас имеются Неизменяемые разделы без Текстов обложки или
какие-то другие комбинации из текстов этих трех категорий,
отредактируйте текст по ситуации.

Если документ содержит значительные отрывки программного кода, мы
рекомендуем Вам одновременно распространять такие отрывки так, чтобы
их можно было использовать как свободное программное обеспечение, на
условиях свободной лицензии на программное обеспечение по выбору,
например,
\href{https://www.gnu.org/licenses/gpl-3.0.en.html}{Генеральной
  публичной лицензии GNU}.
