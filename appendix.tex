%% Copyright (c) 2002, 2010 Sam Williams
%% Copyright (c) 2010 Richard M. Stallman
%% Permission is granted to copy, distribute and/or modify this
%% document under the terms of the GNU Free Documentation License,
%% Version 1.3 or any later version published by the Free Software
%% Foundation; with no Invariant Sections, no Front-Cover Texts, and
%% no Back-Cover Texts. A copy of the license is included in the
%% file called ``gfdl.tex''.

\chapter{Приложение A -- хаки, хакеры и хакерство} \label{Appendix A}

Чтобы лучше понять значение слова \enquote{хакер}, полезно изучить эволюцию этого термина.

\enquote{Новый словарь хакера}, онлайн-сборник жаргона программистов, насчитывает по десятку разных значений для слов \enquote{хак} и \enquote{хакер}. Но в сопроводительном эссе Фил Эгре, хакер МТИ, призывает не обманываться кажущейся гибкостью и многозначностью этих слов. \enquote{У хака есть только одно значение, -- пишет он, -- и оно настолько тонкое и глубокое, что не поддаётся определению}. Ричард Столлман же предлагает своё толкование: \enquote{игривая сообразительность}.

Всё же большинство хакеров сходятся во мнении, что термин этот обязан своим появлением Массачусетскому технологическому институту, точнее, жаргону его студентов начала 50-х годов. В 1990 году музей МТИ составил журнал, посвящённый истории хакерства. В нём говорится, что студенты 50-х годов употребляли слово \enquote{хак} примерно в том же значении, в котором современные студенты употребляют \enquote{фигня}. Например, \enquote{хаком} могли назвать вывешивание хлама из окна общежития или что-то в этом роде, но не злонамеренное, вроде выталкивания студента из окна или порчи институтских статуй. Словом, \enquote{хак} носил дух безобидного, весёлого творчества.

Отсюда пошла отглагольная форма: \enquote{хакерство}. Студент 50-х мог бы назвать \enquote{хакерством} ковыряние в радиоприёмнике или болтовню по телефону целый день напролёт. Опять же, современный представитель молодёжи в этих случаях сказал бы: \enquote{страдал фигнёй}.

Но уже в тех же 50-х годах слово \enquote{хак} приобрело более острый, бунтарский характер. МТИ этого периода был очень конкурентным, и хакерство стало ответом на это расширение культуры конкуренции. Студенты начали \enquote{страдать фигнёй} и откалывать всякие забавные штуки, чтобы выпустить пар и утереть нос администрации студенческого городка, и заодно дать волю творческому поведению, которое подавлялось строгой учебной программой. Институт с его бесчисленными запутанными коридорами и подземными туннелями давал студентам массу возможностей забавляться, игнорируя запертые двери и знаки \enquote{посторонним вход воспрещён}. Шныряющие в подземельях студенты стали называть свои шальные исследования \enquote{туннельным хакерством}. Выше уровня земли такие же возможности давала телефонная сеть института. Благодаря случайным экспериментам и должной осмотрительности студенты научились проворачивать шутливые трюки, называя их \enquote{телефонным хакерством}.

Творческая игра и исследования без ограничений -- вот основа последующих мутаций термина \enquote{хакерство}. В конце 50-х годов студенты МТИ образовали Технический клуб железнодорожного моделирования (Tech Model Railroad Club). Комитет по сигналам и питанию (Signals and Power (S\&P) Committee) этого клуба отвечал за электрические схемы моделей, и был самой настоящей шайкой электротехнических маньяков. Схемы представляли собой сложные наборы реле и переключателей, сделанные по образу и подобию телефонной системы института. Члены клуба управляли своими поездами, набирая команды на телефоне.

Очевидно, что до телефонного хакерства тут было рукой подать. И именно инженеры-электрики железнодорожного клуба дали ему дальнейшее развитие. Если инженер использовал в схеме на одно реле меньше, то это давало ему на одно реле больше в дальнейших играх. Скоро члены S\&P стали называть свою работу по совершенствованию электрических схем \enquote{хакерством}, а себя -- \enquote{хакерами}. Так хакерство превратилось из бестолковой забавы в забаву толковую -- оно увеличивало общую эффективность железнодорожной модели.

Учитывая пристрастие студентов МТИ к сложной электронике, их пренебрежение к запертым дверям и знакам \enquote{посторонним вход воспрещён}, им понадобилось совсем немного времени, чтобы узнать о появлении новой машины в институте. Машина эта, TX-0, была одним из первых коммерческих компьютеров. К исходу 50-х годов вся шайка S\&P переместилась из помещений железнодорожного клуба в диспетчерскую TX-0. Произошла очередная этимологическая мутация -- \enquote{хакерством} перестали называть пайку причудливых электрических схем, теперь это было комбинирование программ за рамками \enquote{официальных} методов и рекомендаций по написанию кода. Благодаря таким манипуляциям с кодом росла производительность и высвобождались дорогостоящие машинные ресурсы. Конечно же, создаваемые хакерами программы не преследовали никакой иной цели, кроме развлечения.

Классический пример хакерства -- создание первой компьютерной видеоигры Spacewar. Она подходит по всем критериям, её на ходу разработали хакеры в начале 60-х годов только ради того, чтобы проводить за нею ночи напролёт. При этом, с точки зрения программирования, эта игра была сплошным новаторством. Также она была совершенно свободна. Хакеры делали игру для себя, так что они не видели причин хранить её в тайне за семью печатями. Напротив, они охотно раздавали её другим программистам. К концу 60-х Spacewar стала любимой игрой программистов всего мира -- тех, кому посчастливилось тогда иметь компьютеры с графическим выводом.

Коллективное создание и общее владение отдалили компьютерное хакерство 60-х от туннельного и телефонного хакерства 50-х годов. Ведь последним занимались в одиночку или маленькими группами. Туннельные и телефонные хакеры изучали строение и деятельность института, но не могли открыто делиться этими знаниями. Компьютерные же хакеры изначально работали в научной сфере с её давними традициями коллективной работы и вознаграждения за обмен информацией. Конечно, хакеры и научные сотрудники не всегда ладили друг с другом, но им удалось развить эффективное сотрудничество, которое впору называть симбиозом.

Хакеры почти совсем не уважали бюрократические правила. Системы безопасности для них были очередной досадной ошибкой, которую нужно поскорее исправить. Так, взлом систем безопасности (но не злонамеренный взлом) стал частью хакерства в 70-х годах, используемым для получения доступа и откалывания различных шуток. Жертва такой шутки обычно говорила что-то вроде: \enquote{похоже, меня хакнули}. Но идея взлома не была центральной в хакерстве. Хакеры с гордостью ломали системы безопасности, но там, где таких систем не было, они ничего не взламывали, предпочитая заниматься другими вещами. Кому нужно ломать то, что никак не портит жизнь?

Мастерство компьютерных хакеров становилось очевидно всем, так что программисты новых поколений, включая Ричарда Столлмана, стремились облачиться в ту же хакерскую мантию. К середине 70-х термин \enquote{хакер} приобрёл оттенок элитарности. Так называли программиста, который пишет код ради удовольствия и добивается высокого мастерства. Признать человека хакером в среде программистов -- лучший способ выразить уважение к нему и подчеркнуть свою принадлежность к касте компьютерных специалистов, подобно тому, как люди искусства говорят о своём собрате: \enquote{он настоящий художник}. Назвать хакером себя -- значит заявить о своих исключительных навыках программирования.

Компьютеры становились всё популярнее, а термин -- всё конкретнее. Вместе с большей определённостью \enquote{компьютерное хакерство} получило дополнительные смысловые оттенки. Хакеры Лаборатории ИИ из МТИ рассказали о своём пристрастии к китайской еде, отвращении к табачному дыму, отказе от алкоголя и наркотиков. Многие члены сообщества принимали эти дополнительные черты хакерства, но не все. Вместе с разрушением коллектива Лаборатории значение этих ценностей сошло на нет. Сегодня большинство хакеров мало чем отличаются от окружающих людей.

По мере того, как хакеры элитных вузов вроде МТИ, Стэнфорда, Карнеги-Меллона обсуждали восхитительные примеры хаков, они также поднимали вопросы своей \enquote{хакерской этики} -- неписаного кодекса поведения хакера. В 1984 году Стивен Леви в своей книге \enquote{Хакеры} привёл 5 основных принципов хакера, выведя их из своих исследований по теме.

В 80-х годах определение хакерства снова претерпело изменения, прежде всего, из-за распространения компьютерных взломов. Большинство таких взломов делались людьми, которые не имели никакого отношения к хакерству в его первоначальном смысле. Однако для полиции и администраторов, которые видят непослушание злом, все они были хакерами, несмотря на одно из правил хакерской этики: \enquote{не навреди людям}. Журналисты стали публиковать статьи, в которых компьютерные взломы вовсю именовали \enquote{хакерством}. И хотя такие авторы, как Стивен Леви, всё ещё обращали внимание публики на настоящий смысл слова \enquote{хакер}, для людей он стал обозначать компьютерного взломщика.

В поздних 80-х многие американские подростки уже имели тот или иной доступ к компьютерам. И обычная для подростков отчуждённость от общества, вдохновлённая искажённым журналистским определением хакерства, вылилась во взломы компьютерных систем -- новую форму битья стёкол и прочего хулиганства. Эти компьютерные хулиганы называли себя хакерами, даже не подозревая о хакерской этике с её неприятием злонамеренности. По мере того, как компьютерные взломы и создание вирусов набирали популярность, \enquote{хакер} стал считаться этаким панком, нигилистом-борцом против Системы, и такой образ привлекал немало людей.

Настоящие же хакеры постоянно протестуют против таких терминологических извращений. Столлман, с его характерным отрицанием некоторых общепринятых вещей, предложил называть компьютерный взлом не \enquote{хакерством}, а \enquote{кракерством}. Различие между этими терминами, впрочем, не принципиальное. Нельзя сказать, что это отдельные виды деятельности, которые никогда не пересекаются. Хакерство и кракерство -- разные проявления одной и той же деятельности, так же как \enquote{низкий} и \enquote{высокий} -- разные проявления человеческого роста.

Хакерство, как правило, не посягает на безопасность, так что это не взлом. Взлом же, как правило, делается ради корысти или из злого умысла, так что это не хакерство. Бывает, что одно и то же действие или занятие можно называть и взломом, и хакерством, но это редкость. Хотя хакерский дух включает неуважение к правилам, большинство хаков не нарушают правил. Взлом это определённо нарушение, но оно необязательно злонамеренно или вредно. Специалисты по компьютерной безопасности различают \enquote{белошляпых} и \enquote{черношляпых} взломщиков -- первые исследуют системы и ищут уязвимости, вторые взламывают системы ради корысти или хулиганства.

Главный принцип -- не желать зла -- связывает современного хакера с его предшественником 50-х годов. И если понятие компьютерного хакерства постоянно менялось, то значение просто хакерства осталось таким же, каким было изначально -- то есть, проделыванием различных шуток и выходок. В конце 2000 года музей МТИ отдал должное старой хакерской традиции института, организовав Зал Хака. В нём выставили ряд фотографий с проделками студентов, включая снимки 20-х годов, на одной из которых запечатлён самодельный макет полицейской машины. В 1993 году студенты МТИ повторили этот хак, водрузив такой макет с настоящими работающими мигалками на главный купол института. На номерном знаке макета было написано IHTFP -- популярное в институте сокращение с уймой значений, самое частое из которых наполнено безысходностью: \enquote{ненавижу это долбаное место} (\enquote{I hate this fucking place}). Однако в 1990 году музей МТИ использовал этот акроним в названии своего журнала об истории хакерства: \textit{The Journal of the Institute for Hacks, Tomfoolery, and Pranks} или \enquote{Институтский журнал хаков, дурачества и шалостей}, дав таким образом пример искусного хака.

\enquote{Культура хакерства ценит простые, изящные решения, что роднит её с культурой науки}, -- написал в 1993 году репортёр \textit{Boston Globe} Рендольф Райан, и его статья также вошла в экспонаты Зала Хака. \enquote{Хак отличается от обычной проделки тем, что тщательно спланирован и проработан с инженерной точностью, в нём заложены изобретательность и остроумие, -- рассказывает Райан, -- неписанные правила гласят, что хак должен быть добродушным, недеструктивным и безопасным. Хакеры нередко помогают убрать все следы своих проделок}.

Ограничить культуру компьютерного хакерства теми же рамками вряд ли возможно. Хотя программные хаки тоже нацелены на изящество и простоту, программная среда даёт меньше возможностей для обратимости действий. Куда легче убрать с крыши полицейскую машину, чем уничтожить идею, особенно ту, чей час пробил.

Слово \enquote{хакер}, что было когда-то смутным понятием из студенческого жаргона, стало лингвистическим бильярдным шаром, который гоняют как хотят ради политических или этических целей. Это слово любят употреблять и сами хакеры, и журналисты, и простые люди. Вряд ли можно предсказать, как его будут использовать в будущем. Но можно решить, как использовать его сейчас. Называть \enquote{хакерами} не компьютерных взломщиков (\enquote{кракеров}), а талантливых программистов, пишущих изящный и мощный код -- значит проявлять уважение к Столлману и остальным упомянутым в книге хакерам, а также помогать сохранить то, что дало нам всем так много хорошего: настоящий хакерский дух.
